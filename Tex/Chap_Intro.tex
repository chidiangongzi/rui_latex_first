\chapter{引言}\label{chap:introduction}

\section{研究背景}

大气气溶胶是悬浮在大气中的液体或者固态颗粒,其中$PM_{10}$是指颗粒物动力学直径小于或等于10 $\mu$ m的颗粒物,$PM_{2.5}$ 是指动力学直径小于或等于2.5 $\mu$ m的颗粒物,又称为细颗粒物体(fine particle matter),是大气的重要组成部分。$PM_{1.5}$ 粒径小,面积大,活性强。在大气中停留时间长、传输距离远,尽管其在大气成分中含量很少,但对空气质量和能见度等有重要的影响。近年来,越来越多的研究证明大气污染对人体的健康都有直接和间接的关系。

\subsection{研究现状}

\citet{唐洁2019三种}探究了不同边界层参数化方案对不同地域的强对流降水具有不同的模拟效果,YSU和MRF方案模拟结果与实况更符合。\citet{张颖龙2017基于}运用WRF-Chem空气质量模式分析了京津冀一次污染过程的时空分布特征以及长距离输送事件,通过源解析分析表明该次污染收天气形势、地形、便阶层特征共同影响。\citet{王瑾2017基于}通过WRF-Chem和HYSPLIT对徐州的4次重污染过程分析表明:北部加强高压,高空有西北气流,徐州位于低压后部、高压前部、高压后部时候,较强北风、东风利于污染物扩散。重污染时边界层高度较低。
