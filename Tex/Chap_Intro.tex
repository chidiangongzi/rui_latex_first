\chapter{引言}\label{chap:introduction}

\section{研究背景}


\subsection{参考文献引用}

参考文献引用过程以实例进行介绍,假设需要引用名为"Document Preparation System"的文献,步骤如下:

1)使用Google Scholar搜索Document Preparation System,在目标条目下点击Cite,展开后选择Import into BibTeX打开此文章的BibTeX索引信息,将它们copy添加到ref.bib文件中(此文件位于Biblio文件夹下)。

2)索引第一行 \verb|@article{lamport1986document,|中 \verb|lamport1986document| 即为此文献的label (\textbf{中文文献也必须使用英文label},一般遵照:姓氏拼音+年份+标题第一字拼音的格式),想要在论文中索引此文献,有两种索引类型:

文本类型:\verb|\citet{lamport1986document}|。正如此处所示 \citet{lamport1986document}; 

括号类型:\verb|\citep{lamport1986document}|。正如此处所示 \citep{lamport1986document}。

\textbf{多文献索引用英文逗号隔开}:

\verb|\citep{lamport1986document, chu2004tushu, chen2005zhulu}|。正如此处所示 \citep{lamport1986document, chu2004tushu, chen2005zhulu}

更多例子如:

\citet{walls2013drought} 根据 \citet{betts2005aging} 的研究,首次提出...。其中关于... \citep{walls2013drought, betts2005aging},是当前中国...得到迅速发展的研究领域 \citep{chen1980zhongguo, bravo1990comparative}。引用同一著者在同一年份出版的多篇文献时,在出版年份之后用
英文小写字母区别,如:\citep{yuan2012lana, yuan2012lanb, yuan2012lanc} 和 \citet{yuan2012lana, yuan2012lanb, yuan2012lanc}。同一处引用多篇文献时,按出版年份由近及远依次标注。例如 \citep{chen1980zhongguo, stamerjohanns2009mathml, hls2012jinji, niu2013zonghe}。

使用著者-出版年制(authoryear)式参考文献样式时,中文文献必须在BibTeX索引信息的 \textbf{key} 域(请参考ref.bib文件)填写作者姓名的拼音,才能使得文献列表按照拼音排序。参考文献表中的条目(不排序号),先按语种分类排列,语种顺 序是:中文、日文、英文、俄文、其他文种。然后,中文按汉语拼音字母顺序排列,日文按第一著者的姓氏笔画排序,西文和 俄文按第一著者姓氏首字母顺序排列。如中 \citep{niu2013zonghe}、日 \citep{Bohan1928}、英 \citep{stamerjohanns2009mathml}、俄 \citep{Dubrovin1906}。

如此,即完成了文献的索引,请查看下本文档的参考文献一章,看看是不是就是这么简单呢?是的,就是这么简单!

不同文献样式和引用样式,如著者-出版年制(authoryear)、顺序编码制(numbers)、上标顺序编码制(super)可在Thesis.tex中对artratex.sty调用实现,详见 \href{https://github.com/mohuangrui/ucasthesis/wiki}{ucasthesis 知识小站之文献样式}

%若在上标顺序编码制(super)模式下,希望在特定位置将上标改为嵌入式标,可使用 \citetns{niu2013zonghe,stamerjohanns2009mathml} 和 \citepns{niu2013zonghe,stamerjohanns2009mathml}。

参考文献索引的更多知识,请见 \href{https://en.wikibooks.org/wiki/LaTeX/Bibliography_Management}{WiKibook Bibliography}。\nocite{*}% 使文献列表显示所有参考文献(包括未引用文献)

BBBBBBBBBBBBBBBBBBBBBBBBBBBBb

CCCCCCCCCCCCCCCCCCCCCCCCCCCCCcccc

\section{dd}\label{sec:system}

EEEEEEEEEEEEEEEEEEEEEEEEEEEEEE


FFFFFFFFFFFFFFFFFFFFFFFFFFFFFFF

\section{h}

LLLLLLL

MMMMMMM

