%---------------------------------------------------------------------------%
%->> Frontmatter
%---------------------------------------------------------------------------%
%-
%-> 生成封面
%-
\maketitle% 生成中文封面
\MAKETITLE% 生成英文封面
%-
%-> 作者声明
%-
\makedeclaration% 生成声明页
%-
%-> 中文摘要
%-
\intobmk\chapter*{摘\quad 要}% 显示在书签但不显示在目录
\setcounter{page}{1}% 开始页码
\pagenumbering{Roman}% 页码符号

本文利用常规站点资料,采用耦合大气化学区域模块的WRF-Chem中尺度数值模式对江苏省一次轻度污染过程进行一次数值模拟过程,通过数值实验及分析对比,研究了边界层对污染物浓度的敏感性,以及该次污染物过程的水平垂直分布特征。更进一步分析该此污染过程形成扩散积累机制。结果表明:该次污染物受动力和热力影响,污染物水平分布主要跟随风场移动,强的垂直风切变有利于污染在垂直方向的扩散,浅薄逆温层有利于低层污染物的堆积,除臭氧外的污染物浓度同边界层高度呈负相关。

\keywords{WRF-CHEM,$PM_{2.5}$,大气扩散}% 中文关键词
%-
%-> 英文摘要
%-
\intobmk\chapter*{Abstract}% 显示在书签但不显示在目录

This paper, by using conventional station data, using the coupled atmospheric chemistry area module WRF - Chem mesoscale numerical model of jiangsu province a mild pollution process in a numerical simulation process, through the numerical experiment and comparative analysis, to study the boundary layer on the sensitivity of the pollutant concentration, and the level of pollutants in the process of vertical distribution characteristics. Further analysis the process of the pollution diffusion accumulation mechanism. The results show that the horizontal distribution of pollutants mainly moves with the wind field under the influence of power and heat, the strong vertical wind shear is conducive to the vertical diffusion of pollution, the shallow inversion layer is conducive to the accumulation of low-level pollutants, and the pollutant concentration except ozone is negatively correlated with the boundary layer height.


\KEYWORDS{WRF-CHEM, $PM_{2.5}$, Atmospheric Diffusion}% 英文关键词
%---------------------------------------------------------------------------%
